\documentclass{article}
\usepackage[left=3cm,right=3cm,top=1.5cm,bottom=2cm]{geometry} % page settings
\usepackage{amsmath} % provides many mathematical environments & tools
\usepackage{amsfonts}
\usepackage{amsthm}

\setlength{\parindent}{0mm}

\begin{document}

\title{Review - Chapter 4: Basic Topology}
\author{Parker Hyde}
\date{\today}
\maketitle

\newtheorem{theorem}{Theorem}
\newtheorem{definition}{Definition}
\newtheorem{exmp}{Example}
\newtheorem{proposition}{Proposition}
\newtheorem{lemma}{Lemma}
\newtheorem{corollary}{Corollary}
\newtheorem{remark}{Remark}

\section*{4.3 Compact Sets}

In section 4.2 we established that every infinte sequence in a bounded sets \( S \) has
a convergent subsequence. In particular, we used this to show that bounded infinite sets,
which 'do' have infinite sequences, must have accumulation points. This was the theorem
of Bolzano-Weierstrass.
\smallbreak
It turns out that the converse is also true. We verify this in the following lemma. 

\begin{lemma}
    If a set \( S \subset R \) has the property that every sequence \( a_j \) in \( S \) 
    has a convergent subsequence \( a_{j_k} \), then \( S \) is bounded.
\end{lemma}
\begin{proof}
    Suppose for a contradiction that \( S \) is not bounded. Then we can construct
    a sequnce \( \{a_j\} \) where each \( a_j > j  \ \forall j \in \mathbb{N} \). Now
    consider any subsequence \( a_{j_k} \). Let's take the long road and show that this 
    subsequence cannot be Cauchy. \\
    Let \( \epsilon = 1 > 0 \). Now let \(N \in \mathbb{N}\) and consider \( a_{j_k}, j > N \).
    Then \( m = \lceil a_{j_k} \rceil + 1 \in \mathbb{N}\) so there is some
    subsequence element \( a_{j_k^\prime} > m\). Thus \( | a_{j_k^\prime} - a_{j_k} | > 1 = \epsilon \).
    Hence we can't find an  \( N \) satisfying the \( \epsilon - N \) definition for \( \{a_{j_k}\} \). 
    \( a_{j_k} \) is not Cauchy. \smallbreak
    This is a contradiction so \( S \) must be bounded.
\end{proof}
We now have two independent identities concerning bounded and closed sets respectively.
\medbreak
1) A set \( S \) is \textbf{bounded} if and only if every sequence \( a_j \) in \( S \) has a convergent
subsequence \( a_{j_k} \).

2) A set \( S \) is \textbf{closed} if and only if every Cauchy sequence in \( S \) 
converges to a limit point \( \alpha \in S \).
\medbreak

It follows that a set is closed and bounded if and only if the two RHS statments in the
identities above are true. But we can get a simpler identity.

\begin{lemma}
    Let \( S \subset R \) be a set with the property that every sequence \( a_j \in S \) 
    has a subsequence that converges to a limit point \( \alpha \in S \). Then S satisfies
    the RHS statements of (1) and (2).
\end{lemma}

\begin{proof}
    The RHS of (1) is trivial. To prove (2), let \( a_j \) be a Cauchy sequence in \( S \).
    Then \( a_j \) converges to a limit point \( x \). By hypothesis, \( a_j \) 
    has a subsequence \( a_{j_k} \) that converges to \( \alpha \in S \). But this
    means \( x = \alpha \in S \).
\end{proof}
It's pretty easy to see that the converse of lemma 2 holds so we can skip it. Just
notice that applying (2) to the subsequence produced by (1) yields the desired property.
\medbreak
Thus we get

\begin{theorem}[Heine-Borel]
    A set \( S \subset R \) is \textbf{closed} and \textbf{bounded} if and only if
    every sequence in \( S \) has 
    a subsequence that converges to a limit that is also in \( S \).
\end{theorem}

This theorem motivates the following defintion. 

\begin{definition}
    A set \( S \subset R \) is called \textbf{compact} if every sequence in \( S \) has 
    a subsequence that converges to a limit that is also in \( S \).
\end{definition}

From Theroem 1 (Heine-Borel), we get for free that a set is \textit{compact} if and 
only if it is closed and bounded.




\end{document}
