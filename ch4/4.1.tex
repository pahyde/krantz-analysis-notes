\documentclass{article}
\usepackage[left=3cm,right=3cm,top=1.5cm,bottom=2cm]{geometry} % page settings
\usepackage{amsmath} % provides many mathematical environments & tools
\usepackage{amsthm}

\setlength{\parindent}{0mm}

\begin{document}

\title{Review - Chapter 4: Basic Topology}
\author{Parker Hyde}
\date{\today}
\maketitle

\newtheorem{theorem}{Theorem}

\section*{4.1 Open and Closed Sets}

\subsection*{Interval notation}
For real numbers \( a \le b \), we define

 \begin{align*}
     (a,b) &= \{x \in R : a < x < b\}\\
     [a,b] &= \{x \in R : a \le x \le b\} \\
     [a,b) &= \{x \in R : a \le x < b\} \\
     (a,b] &= {...}
\end{align*}

These are simplest open, closed, and half-open (half-closed, clopen) sets
everyone knows.

\newtheorem{definition}{Definition}
\begin{definition}[open sets]
    A set \( U \) is \textbf{open} if \( \forall x \in U \) we 
    can find an \( \epsilon > 0 \) so that  \( N_\epsilon(x) \subset U \) 
\end{definition}

Basically the set is open if all points are "padded" by other points in set. 
For any \( x \in (0, 1) \) there's always an infintude of points surrounding that  \( x \).

\subsection*{How to prove a set \( U \) is open}

The general strategy is to fix some \( x \in U \) and then find an \( \epsilon \) 
so that  \( N_\epsilon(x) \subset U\).

\newtheorem{exmp}{Example}
\begin{exmp}
    Show that the set \( U = \{x \in R : |x - m| < d \}\) is open
\end{exmp}

We need to find the appropriate \( \epsilon \). \\
Start by fixing \( x \in U \). If \( |x - m| < d \) then 
\( d - |x - m| > 0 \). This just says that the maximum allowed distance \( d \) 
is greater than whatever distance  \( x \) actually is from \( m \). \\
So lets pick this difference for \( \epsilon \). We set \( \epsilon = d - |x - m| \).  \\
Whatever distance \( x \) is from  \( m \), adding less than \( \epsilon \)
distance will never take  \( x \) more than distance \( d \) from m.
So now we just use this info to prove it.

\begin{proof}
    Fix \( x \in U \) and set \( \epsilon = d - |x-m| > 0\).
    Let \( t \in N_\epsilon(x) \). Then \(|t-x| < \epsilon\) and 
    \begin{align*}
        |t-m| &= |(t-x)+ (x-m)| \\
              &\le |t-x| + |x-m|  \\
              &< d - |x-m| + |x-m| \\
              &= d \\
              &\implies t \in U
    \end{align*} 
    Hence \( N_\epsilon(x) \subset U \).  Get fucked, we done.
\end{proof}
Moving on.. \\
Things we are about to cover: 
\medbreak
1) The (possibly infinite) union of open sets is \( \rightarrow \) open \\
2) The finite intersection of open sets is \( \rightarrow \) open. Not necessarily true for infinite (think \( (-\frac{1}{j}, \frac{1}{j}) \)). \\
3) The (possible infinite) intersection of closed sets is \( \rightarrow \) closed \\
4) The finite union of of closed sets is \( \rightarrow \) closed.
\medbreak
Infinite intersections do weird things to open sets. \\
Infinite unions do weird things to closed sets.

\pagebreak
\begin{theorem}
    If \( U_\alpha \) are open sets (possibly denumerable or uncountable), then
    \[
        U = \bigcup_{\alpha \in A} U_\alpha
    \] 
    is open.
\end{theorem}
How do we go about proving this. Again for any \( x \in U \) we need to produce
\( N_\epsilon(x) \subset U \). In this case we can just borrow an \( N_\epsilon \) 
whatever  \( U_\alpha \)  our \( x \) comes from.
\begin{proof}
    Fix \( x \in U \). Then \( x \in U_\alpha \) for some \( \alpha \).
    \( U_\alpha \) is open so there exists an \( N_\epsilon(x) \subset U_\alpha \subset U \).\\
    \( U \) is open. come get some little bitch.
\end{proof}

\begin{theorem}
    \( U_1, U_2, U_3, ..., U_k \) are open sets, then 
    \[
        V = \bigcap_{j=1}^k U_j
    \] 
    is open.
\end{theorem}

This time we can't just take some \( N_\epsilon(x) \) from \( U_j \), assume it'll be a subset of
\( V \) and call it a day. The intersection with other sets may subtract away some of our 
chosen  \( N_\epsilon(x) \). We have to pick the 'right' \( N_\epsilon(x) \).
\begin{proof}
    Fix \( x \in V \). Then \( x \in U_j\), \( \forall j = 1, 2, ..., k \).
    Each of these sets are open so they all have corresponding 
    neighborhoods \( N_{\epsilon_j}(x) \). Let \( \epsilon = \text{min}\{ \epsilon_j : j = 1, 2, ..., k \} \). \\
    Then  \( N_\epsilon(x) \subset U_j\) for \( j = 1, 2, ..., k \)  
    \[ \implies N_\epsilon(x) \subset V  \].
\end{proof}

Again this is not necessarily true. For a situation like 
\[
    \bigcap_{j=1}^\infty = (- \frac{1}{j}, \frac{1}{j}) \quad \text{or even} \quad
    \bigcap_{j=1}^\infty = (0, 1 + \frac{1}{j})
\] 
we get closed and clopen sets respectively. \( 1 \) is in the second set but a
suitable \( N_\epsilon(1) \) doesn't exist.

\begin{theorem}
    \( U \subset R \) is a nonempty open set. Then
    for either finitiely many or countably many (\( k = \infty \)) pairwise disjoint
    open intervals \( I_j \).
    \[
        U = \bigcup_{j=1}^k I_j 
    \] 
\end{theorem}
\begin{proof}
    This is the proof by equivalence relation. Setting \( a \cong b \) if \( (a,b) \subset U \)
    creates a partion with equivalence classes \( I_\alpha \). Being equivalence classes 
    means their union equals \( U \). We just need to show the classes are open intervals.
    Clearly they're intervals. Let \( x \in I_\alpha \). Then  \( x \in U \) so we have a
    \( N_\epsilon(x) \). We constructed the equivalence relation so that 
    \( N_\epsilon(x) \subset I_\alpha \). All the points in \( N_\epsilon(x) \) would
    have to be in \( I_\alpha \).
\end{proof}

Basically, open intervals are the building blocks of open sets. If you have an open set, you
can build it from a union of open intervals. This is trivial for open sets that are 
intervals themselves. 
\smallbreak
The contrapositive is interesting here. If you can't set \( U \) equal to a union of open 
intervals then \( U \) isn't an open set.

\pagebreak
\subsection*{Closed Sets}

\begin{definition}
    \( F \subset R \) is closed in the complement \( R \setminus F \) is open.
\end{definition}

Immediately we should notice the potential contradiction. The sets \( \emptyset \)
and \( R \) are both open by definition (vacuously true for \( \emptyset \)). 
They are mutual complements so they sould both also be closed. In fact
they're both closed and open.. At least that's what I read on math stack.

\begin{exmp}
    The set \( B = \{\frac{1}{j} : j = 1, 2, 3, ...\} \cup \{0\}\) is closed because it's 
    complement is the union of open sets given by
    \[
        (-\infty, 0) \ \cup  \ \bigcup_{j=1}^\infty (\frac{1}{j+1}, \frac{1}{j}) \ \cup  \ (1, \infty)
    \]
\end{exmp}

\begin{theorem}
    
\end{theorem}


\end{document}
