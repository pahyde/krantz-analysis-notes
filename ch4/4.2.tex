\documentclass{article}
\usepackage[left=3cm,right=3cm,top=1.5cm,bottom=2cm]{geometry} % page settings
\usepackage{amsmath} % provides many mathematical environments & tools
\usepackage{amsthm}

\setlength{\parindent}{0mm}

\begin{document}

\title{Review - Chapter 4: Basic Topology}
\author{Parker Hyde}
\date{\today}
\maketitle

\newtheorem{theorem}{Theorem}
\newtheorem{definition}{Definition}
\newtheorem{exmp}{Example}
\newtheorem{proposition}{Proposition}

\section*{4.2 Points in Open and Closed Sets}

The Krantz book uses the terms \textit{'limit point'} and \textit{'accumulation point'}
interchangebly. Let's define these terms and prove they're the same thing.


\begin{definition}[limit point]
    A point \( x \in R \) is a \textbf{limit point} of a set \( S \subset R \) 
    if \( \forall \epsilon > 0\),  \( N_\epsilon(x) \) contains an element of 
     \( S \) other than \( x \).
\end{definition}
\begin{definition}[accumulation point]
    A point \( x \in R \) is a \textbf{accumulation point} of a set \( S \subset R \) 
    if \( \forall \epsilon > 0\),  \( N_\epsilon(x) \) contains infinitely many elements 
    of \( S \).
\end{definition}

Basically we consider every \( N_\epsilon(x) \) to determine what kind of point
\( x \) is for a set \( S \subset R \).
\begin{align*}
    1) & \text{ if every } N_\epsilon(x) \text{ contains a point other than } x \text{ in } S \rightarrow x \text{ is a limit point}\\
    2) & \text{ if every } N_\epsilon(x) \text{ contains an infinite number of points} \text{ in } S \rightarrow x \text{ is an accumulation point}
\end{align*}

\begin{proposition}
   A point \( x \in S \subset R \) is a limit point of \( S \) iff it's an
   accumulation point of \( S \).
\end{proposition}

Proving accumulation point \( \implies \) limit point seems pretty simple. 
Any \( N_\epsilon(x) \) of an accumulation point contains infinitely many points in  \( S \). 
In particular it contains at least 2 points in \( S \). So it contains a point in \( S \)
other than  \( x \).
\medbreak
Alright lets try the other direction. We have some \( N_\epsilon(x) \) for a limit point
\( x \) and we need to show it satisfies the requirements for an accumulation point. 
\( x \) is a limit point so \( N_\epsilon(x) \) contains some  \( s_1 \ne x \in S\).
Cool, we have one point. Infinitely many to go. But this is actually pretty easy right?
We can just choose \( s_2 \ne x \in S \) from a smaller \( N_{\epsilon^\prime}(x) \) 
where \( \epsilon^\prime \) is set to \( |s_1 - x| \). We can do this infinitely many times 
and always get a new \( s_n \) becaues \( x \) is a limit point. Thus we get an infinite number
of \( s_1 \ne x, s_2 \ne x, ... \in S \) so \( x \) is an accumulation point. 
\smallbreak
Other sources have a variety of definitions for these terms. So this result just 
follows from our particular definitions. Moving on...

\subsection*{boundary points, interior points, isolated points}

\begin{definition}[boundary point]
    \( b \in R \) is a \textbf{boundary point} of \( S \subset R \) if every \( N_\epsilon(b) \)
    contains points in \( S \) and points in \( R \setminus S \).
    We denote the set of boundary points for \( S \) as \( \partial S \).
\end{definition}

For the most part, boundary points are what we expect them to be. You can perturb a boundary
point in the appropriate direction and it will no longer be in \( S \).
\medbreak
Boundary points may or may not be in the set \( S \). We'll see in a second that 
\smallbreak
1) closed sets - contain all their boundary points \\
2) open sets - contain none of their boundary points
\medbreak
Oh this is pretty interesting. The boundary set of \( Q \),  \( \partial Q \), is the 
entire real line. This makes sense because any neighborhood around a rational 
contains infinitely many rational and irrational numbers.

\pagebreak

\begin{definition}[interior point]
    A point \( s \in S \subset R \) is an \textbf{interior point} of \( S \) if
    there is an \( N_\epsilon(s) \subset S \).
\end{definition}

From this definition we see that open sets require \textit{all} points to interior points.

\begin{definition}[isolated point]
    A point \( t \in S \subset R \) is an \textbf{isolated point} if there is an
    \( N_\epsilon(t) \) such that \( N_\epsilon(t) \cap S = \{t\} \)
\end{definition}

\begin{proposition}
    Each point of \( S \subset R \) is either an interior point or a boundary point.
\end{proposition}

\begin{proof}
    let \( x \in S \). If \( x \) is an interior point then we're done. Suppose \( x \) 
    is not an interior point. Then all \( N_\epsilon(x) \) contain points in \( R \setminus S \).
    Also  \( N_\epsilon(x) \) contains \( x \in S \). So \( x \) contains points in both 
    \( S \) and \( R \setminus S \). \( x \) is a boundary point.
\end{proof}

Quick remark. interior points are a special class of boundary points. Also accumulation 
point can either be interior or boundary points but never isolated.

\begin{proposition}
    The boundary \( \partial S \) of a set \( S \subset R \) is also the boundary of
    \( R \setminus S \).
\end{proposition}

\begin{proof}
    \begin{align*}
        b \in \partial S  &\iff \text{there exsits }  N_\epsilon(b) \text{ containing points in } S \text{ and  } R \setminus S \\
                          &\iff \text{there exsits }  N_\epsilon(b) \text{ containing points in } R \setminus S \text{ and  } S \\
                            &\iff b \in \partial R \setminus S
    \end{align*}
\end{proof}

This is trivial if you think about it. The definition of \( \partial A \) and \( \partial A^c \)
is identical no matter the set \( A \).

\end{document}
