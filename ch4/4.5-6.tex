\documentclass{article}
\usepackage[left=3cm,right=3cm,top=1.5cm,bottom=2cm]{geometry} % page settings
\usepackage{amsmath} % provides many mathematical environments & tools
\usepackage{amsthm}

\setlength{\parindent}{0mm}

\begin{document}

\title{Review - Chapter 4: Basic Topology}
\author{Parker Hyde}
\date{\today}
\maketitle

\newtheorem{theorem}{Theorem}
\newtheorem{definition}{Definition}
\newtheorem{exmp}{Example}
\newtheorem{proposition}{Proposition}
\newtheorem{lemma}{Lemma}
\newtheorem{corollary}{Corollary}
\newtheorem{remark}{Remark}

\section*{4.5 Connected and Disconnected Sets}

Different books give different definitions for connected and disconnected sets.
Abbott and Rudin both formulate a definition in terms of \textit{'separated'}
sets and Krantz gives a 3 part definition in terms of open sets.
As usual we'll give these definitions and then feel the need to prove their
equivalence.
\smallbreak

First, the Abbott/Rudin defition. We'll use the notation \( \overline{A} \) to denote
the closure of \( A \). 

\begin{definition}[Disconnected Sets - A/R]
    Two nonempty sets \( A,B \subset R \) are separated if \( \overline{A} \cap B = \emptyset \) 
    and \( A \cap \overline{B} = \emptyset \). A set \( E \subset R \) is \textbf{disconnected} if 
    it can be written as the union \( E = A \cup B \) of separated sets \( A \) and \( B \).
\end{definition}

Krantz is harder to parse.

\begin{definition}[Disconnected Sets - Krantz]
    A set \( S \subset R \) is \textbf{disconnected} if it's possible to find open sets
    \( U \) and \( V \) such that 
    \begin{align}
        &U \cap S \ne \emptyset, V \cap S \ne \emptyset \\
        &(U \cap S) \cap (V \cap S) = \emptyset \\
        &S = (U \cap S)  \cup (V \cap S)
    \end{align}
\end{definition}

Intuitively, the Krantz definition is a bit more transparent if we substitute \( A = U \cap S \)
and  \( B = V \cap S \). This gives the simpler looking requirements
\begin{align*}
        &A \ne \emptyset, B \ne \emptyset\\
        &A \cap B = \emptyset\\
        &S = A \cup B.
\end{align*}
So basically we need disjoint nonempty sets \( A,B \subset S \) whose union gives \( S \).
But we must also recall that \( A = U \cap S \) and \( B = V \cap S \) for open sets \( U \) 
and \( V \). Thus, there must be at least one point 'between' \( A \) and \( B \). This 
is clear if \( A \) and \( B \) end up both being open sets. If one of them is closed, say \( B \),
then it was formed by an open set \( V \) with \( B \subset V \). \( A \) and \( B \) cannot share
some boundary point \( b \in B \). This would require that \( A \cap (B = V \cap S) \ne \emptyset \)
because \( V \) would necessarily contain points in \( A \).

\begin{proof}

\end{proof}


\end{document}
