\documentclass{article}
\usepackage[left=3cm,right=3cm,top=1.5cm,bottom=2cm]{geometry} % page settings
\usepackage{amsmath} % provides many mathematical environments & tools
\usepackage{amsthm}

\setlength{\parindent}{0mm}

\begin{document}

\title{Review - Chapter 5: Limits and Continuity of Functions}
\author{Parker Hyde}
\date{\today}
\maketitle

\newtheorem{theorem}{Theorem}
\newtheorem{definition}{Definition}
\newtheorem{exmp}{Example}
\newtheorem{proposition}{Proposition}
\newtheorem{lemma}{Lemma}
\newtheorem{corollary}{Corollary}
\newtheorem{remark}{Remark}

\section*{5.2 Continuous Functions}

\begin{definition}
    Let \( f \) be a real-valued function with domain \( E \subset R \). Fix
    an accumulation point of \( E \), \( p \in E \).
    Then \( f\) is continuous at \( p \) if 
    \[
        \lim_{x \to p} f(x) = f(p)
    \] 
\end{definition}

Notice that we need \( p \) in the domain \( E \) to even consider continuity.

\begin{exmp}
    The function 
    \[
        h(x) = \begin{cases}
            sin(\frac{1}{x}) & x \ne 0 \\
            1 & x = 0
               \end{cases}
    \] 
    is discontinuous at \( x = 0 \) because \( \lim_{x \to 0} h(x) \) doesn't exist. Thus
    it can't possibly equal \( h(p)\).
\end{exmp}

\begin{proof}
   proof by Proposition 2 from 5.1. We end up with two sequences whos images converge to 
   different values. This can't happen since every sequence image should converge to the 
   same limit.
\end{proof}


\begin{theorem}
    Let \( f \) and \( g \) be functions with domain \( E \) and let \( p \in E\) be 
    an accumulation point of \( E \). If \( f \) and \( g \) are continuous at \( p \) 
    then so are
    \[
        (f+g), (f*g), \text{ and  } \left(\frac{f}{g}\right) \  (\text{provided  }g(p) \ne 0).
    \] 
\end{theorem}

\begin{proof}
    \( \lim_{x \to p} (f + g)(x) =  \lim_{x \to p} f(x) + \lim_{x \to p} g(x) = f(p) + g(p) = (f + g)(p)\).
\end{proof}

The other two proofs are essentially identical. Here's another extension of a 
result given in section 5.1. 

 \begin{proposition}
    \( f \) is a function with domain \( E \) and \( p \in E \) is an accumulation point
    of \( E \). Then
    \[
        \lim_{x \to p} f(x) = f(p)
    \] 
    if and only if
    \[
        a_j \subset E \text{ and } \lim_{j \to \infty} a_j = p \implies \lim_{j \to \infty} f(a_j) = f(p).
    \] 
\end{proposition}
We knew this was true for any limit point, \( \ell \), from section 5.1 proposition 2.
This is just the special case where \( \ell = f(p) \).  
\smallbreak
Restating this fact in the following form will be useful in a second.

\begin{corollary}
    If \( f \) is continuous at \( p \) and \( \lim_{j \to \infty} a_j = p \), then
     \[
         \lim_{j \to \infty} f(a_j) = f\left(\lim_{j \to \infty} a_j\right) 
    \] 
\end{corollary}
Cool, now we can apply it in the next proposition.  
\pagebreak

\begin{proposition}
    Let \( g: D \to E \) and \( f: E \to F \) and suppose \( p \in D \) is an accumulation
    point of \( D \). Assume \( g \) is continuous at \( p \) and \( f \) is continuous at
    \( g(p) \). Then \( f \circ g \) is continuous at \( p \).
\end{proposition}
\begin{proof}
    Let \( a_j \) be a sequence such that \( \lim_{j \to \infty} a_j = p \). Then 
    \begin{align*}
        \lim_{j \to \infty} f \circ g(a_j) &= \lim_{j \to \infty} f(g(a_j)) \\
                                           &= f(\lim_{j \to \infty} g(a_j)) \\
                                           &= f( g( \lim_{j \to \infty} a_j)) \\
                                           &= f( g(p)) \\
                                           &= f \circ g(p) 
    \end{align*}
\end{proof}

This proof is probably best read from bottom to top. We need to be careful when 
assuming \( f \) is continuous at \( \lim_{j \to \infty} g(a_j) \). 

\begin{remark}
    to-do. Discuss limits of composite functions. piecewise and constant function 
    counterexample
\end{remark}

\begin{definition}[inverse image]
    \( f \) is a function with domain \( E \) and \( W \) is any set of real numbers.
    Then 
    \[
        f^{-1}(W) = \{ x \in E : f(x) \in W \}
    \] 
    is the \textbf{inverse image} of \( W \) under \( f \).
\end{definition}

\begin{theorem}
    \( f \) with domain \( E \) is continuous \( \iff \) the inverse image of any open set
    under \( f \) is the intersection of \( E \) with an open set.
    If \( E \) is open, then \( f \) is continuous \( \iff \) the inverse image of every open
    set under \( f \) is open.
\end{theorem}

\begin{proof}
    to-do.
\end{proof}

\begin{corollary}
    \( f \) with domain \( E \) is continuous \( \iff \) the inverse image of any closed set
    under \( f \) is the intersection of \( E \) with a closed set.
    If \( E \) is closed, then \( f \) is continuous \( \iff \) the inverse image of every closed
    set under \( f \) is closed.
\end{corollary}

\begin{proof}
    to-do.
\end{proof}


\end{document}
